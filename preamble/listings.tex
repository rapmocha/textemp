% ソースコードを書く
% シンタックスハイライトもしてくれる
% LaTeX/Source Code Listings - Wikibooks, open books for an open world
% https://en.wikibooks.org/wiki/LaTeX/Source_Code_Listings

\usepackage{listings}
\definecolor{mygreen}{rgb}{0,0.6,0}
\definecolor{mygray}{rgb}{0.5,0.5,0.5}
\definecolor{mymauve}{rgb}{0.58,0,0.82}

\lstset{
    % choose the background color; you must add \usepackage{color} or \usepackage{xcolor}
    backgroundcolor=\color{white},
    % the size of the fonts that are used for the code
    % basicstyle=\ttfamily,
    basicstyle=\ttfamily\footnotesize,
    % sets if automatic breaks should only happen at whitespace
    breakatwhitespace=false,
    % breaklinesで改行した時のインデント量.
    breakindent=20pt,
    % sets automatic line breaking
    breaklines=true,
    % sets the caption-position to bottom
    captionpos=,
    % 書体による列幅の違いを調整するか.
    columns=fixed,
    % comment style
    commentstyle=\color{mygreen},
    % if you want to delete keywords from the given language
    deletekeywords={...},
    % if you want to add LaTeX within your code
    escapeinside={\%*}{*)},
    % lets you use non-ASCII characters; for 8-bits encodings only, does not work with UTF-8
    extendedchars=true,
    % adds a frame around the code
    frame=tblr,
    % frame=single,
    % keeps spaces in text, useful for keeping indentation of code (possibly needs columns=flexible)
    keepspaces=true,
    % keyword style
    keywordstyle=\color{blue},
    % the language of the code
    language=Octave,
    % if you want to add more keywords to the set
    otherkeywords={*,...},
    % where to put the line-numbers; possible values are (none, left, right)
    numbers=none,
    % how far the line-numbers are from the code
    numbersep=7pt,
    % the style that is used for the line-numbers
    % numberstyle=\tiny\color{mygray},
    numberstyle=\color{mygray},
    % if not set, the frame-color may be changed on line-breaks within not-black text (e.g. comments (green here))
    rulecolor=\color{black},
    % show spaces everywhere adding particular underscores; it overrides 'showstringspaces'
    showspaces=false,
    % underline spaces within strings only
    showstringspaces=false,
    % show tabs within strings adding particular underscores
    showtabs=false,
    % the step between two line-numbers. If it's 1, each line will be numbered
    stepnumber=1,
    % string literal style
    stringstyle=\color{mymauve},
    % sets default tabsize to 2 spaces
    tabsize=4,
    % show the filename of files included with \lstinputlisting; also try caption instead of title
    title=\lstname,
    xleftmargin=10pt,
    xrightmargin=10pt
}

% code環境
% figureやtableとおなじようなfloatにする。
% [TeXの記憶(6)—独自のフロート環境を作る | 寝坊した](http://oversleptabit.com/?p=119)
\makeatletter
\newcounter{code}
\newcommand{\codename}{リスト}
\def\fps@code{tbp}
\def\ftype@code{8}
\def\ext@code{lof}
\def\fnum@code{\codename\nobreak\thecode}
\newenvironment{code}%
               {\@float{code}}%
               {\end@float}
\newenvironment{code*}%
               {\@dblfloat{code}}%
               {\end@dblfloat}
\makeatother

